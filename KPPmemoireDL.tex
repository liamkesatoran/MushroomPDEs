\documentclass[12pt]{extarticle}
%Some packages I commonly use.
\usepackage[english]{babel}
\usepackage{graphicx}
\usepackage{framed}
\usepackage[normalem]{ulem}
\usepackage{amsmath}
\usepackage{amsthm}
\usepackage{amssymb}
\usepackage{amsfonts}
\usepackage{enumerate}
\usepackage{physics}
\usepackage[utf8]{inputenc}
\usepackage[top=1 in,bottom=1in, left=1 in, right=1 in]{geometry}

%A bunch of definitions that make my life easier\
\newcommand{\dt}{\partial_t}
\newcommand{\Tl}{\frac{T}{\lambda}}
\setlength{\columnseprule}{1 pt}


\title{Étude par perturbation du modèle avec Mémoire du champignon (sans le terme du type Keller Segel)}
\author{Liam Toran} 


\begin{document}

\maketitle

\section{Equation KPP avec Memoire}
On a le modele suivant: 
\begin{equation} \left\{
                \begin{array}{ll}
                   \dt\mu -\frac{T}{\lambda}\Delta\mu = f(C)(\mu + \rho) -\mu\rho\\
                 \dt\rho=  F_0 \mu \\
                  \dt C = -b\rho C
                \end{array}
              \right.
\end{equation} 
où $f(0)=0$ et $f$ est positive.\\
On recherche des solutions en onde plane, on pose $c$ la vitesse d'onde et $\xi = x - ct$. \\
\begin{equation} \left\{ \begin{array}{ll} -c \mu'-\Tl\mu''=f(C)(\mu+\rho)-\mu\rho \\ -c\rho' = F_0\mu  \\C'=\frac{b\rho C}{c} \end{array}\right.
\end{equation}
Nos etats stationaires sont $(\mu,\rho,C) = \left\{ \begin{array}{ll} (0,0,C_0) \\
 (0,\rho_\infty,0) \end{array} \right.$ 
\subsection{Au voisinage de $(0,0,C_0)$}
Au voisinage de $(0,0,C_0)$ on a, en posant $f(C_0)=f_0$:
\begin{equation} \left\{ \begin{array}{ll} -c \mu'-\Tl\mu''=f_0(\mu+\rho) \\ -c\rho' = F_0\mu   \end{array}\right.
\end{equation} ce qui devient  \begin{equation} \rho''' +\frac{c\lambda}{T}\rho''+\frac{f_0\lambda}{T}\rho'-\frac{\lambda F_0f_0}{Tc}\rho =0 \end{equation} de polynôme caracteristique \begin{equation} P(X)= X^3 +\frac{c\lambda}{T}X^2+\frac{f_0\lambda}{T}X-\frac{\lambda F_0f_0}{Tc} \end{equation}
Pour $c<0$,   $P(0)>0$ donc P a une racine negative $r_1$ .\\
Pour que P ait deux autres racines reelles $r_3>r_2>r_1$ il faut (condition necessaire et suffisante) que P' s'annule deux fois et que le discriminant $\Delta$ de P soit positif.
\subsubsection{Première condition: P' a deux annulations:}
$P'(X)= 3X^2+ 2\frac{c\lambda}{T}X+ \frac{f_0\lambda}{T}$ a pour discriminant $\Delta'=4(\frac{\lambda}{T})^2(c^2-3 \Tl f_0)$ ce qui donne la condition \begin{equation} \label{eq:condition_P'}
	\boxed{c^2 >3\Tl f_0
	}
\end{equation}
\subsubsection{Deuxième condition: $\Delta>0$:}
Pour $P=aX^3 +bX^2 + cX + d$ on a $\Delta= b^2c^2 +18abcd-27a^2d^2 -4ac^3 -4b^3d$ ce qui dans notre cas donne
\begin{align*}
	\Delta=\frac{\lambda^4}{T^4}f_0^2c^2 -18 \frac{\lambda^3f_0^2F_0}{T^3}-27 \frac{\lambda^2 F_0^2 f_0^2}{T^2c^2} - 4 \frac{f_0^3\lambda^3}{T^3}+4 \frac{\lambda^4F_0f_0c^2}{T^4} \\ = c^2 \frac{\lambda^4f_0(f_0+4F_0)}{T^4}- \frac{\lambda^3f_0^2(18F_0+4)}{T^3} -\frac{27\lambda^2F_0^2f_0^2}{T^2}* \frac{1}{c^2}\\ 
=	\frac{\lambda^4f_0}{T^4c^2}[(f_0+4F_0)c^4-\frac{Tf_0(18F_0+4)}{\lambda} c^2 - 27 \frac{T^2F_0^2f_0}{\lambda^2}] \end{align*}
On est revenu à étudier le signe du polynôme en $c^2$ \begin{equation}
	D(c^2)=(f_0+4F_0)c^4-\frac{Tf_0(18F_0+4)}{\lambda} c^2 - 27 \frac{T^2F_0^2f_0}{\lambda^2}
\end{equation} 
de discriminant $d$:
\begin{align*}
	d=\Big(\frac{Tf_0(18F_0+4)}{\lambda} \Big)^2 +108(f_0+4F_0)\frac{T^2F_0^2f_0}{\lambda^2} \\ 
	= \frac{T^2f_0}{\lambda^2}(f_0(18F_0+4)^2+108(f_0+4F_0)F_0^2) >0
\end{align*}
On obtient donc la condition sur la positivite de $\Delta$: 
\begin{equation}\boxed{
	c^2> \frac{Tf_0(18F_0+4)+T\sqrt{f_0(f_0(18F_0+4)^2+108(f_0+4F_0)F_0^2)}}{2\lambda(f_0+4F_0)}
	}\label{eq:condition_Delta}
\end{equation}
\subsubsection{Signe des racines}
On sait déjà que $r_3<0$. Comme $r_1r_2r_3<0$, on remarque que $r_2$ et $r_1$ sont du meme signe.\\
De plus $P'$ a un axe de symetrie $X=-\frac{c\lambda}{3T}>0$ car $c<0$ donc $P$ atteint un minimum local (forcement négatif) en un point positif donc $P$ a une racine positive.\\
On en deduit $r_1>r_2>0$: \\ 
Sous les conditions \eqref{eq:condition_P'} et \eqref{eq:condition_Delta}, $P$ a deux racines positives et une négative.\newpage
\subsection{Au voisinage de $(0,\rho_\infty,0)$} 
Autour de $(0,\rho_\infty,0)$:
\begin{equation} \left\{ \begin{array}{ll} -c \mu'-\Tl\mu''=f(C)\rho_\infty-\mu\rho_\infty\\C'=\frac{b\rho_\infty C}{c} \end{array}\right.
\end{equation}
la deuxième ligne donne \begin{equation}
C = K\exp(\frac{b\rho_\infty}{c}t )
\end{equation}
et la première est une EDO d'ordre deux  en $\mu$ avec terme source $f(C)\rho_\infty$ de polynôme caracteristique: \begin{equation}
	Q(X)=X^2+\frac{c\lambda}{T}X-\frac{\rho_\infty\lambda}{T}
\end{equation} 
qui possède toujours deux racines: une négative et une positive.
\end{document}

