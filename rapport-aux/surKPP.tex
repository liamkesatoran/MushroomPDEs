%\documentclass[11pt]{extarticle}
%Some packages I commonly use.

%A bunch of definitions that make my life easier\

%\setlength{\columnseprule}{1 pt}
%\begin{document}


\section{L'équation de Fisher ou KPP}
\subsection{Préliminaire}
Notre point de départ est l'équation de diffusion:\begin{equation}\dt u = \Delta u  \end{equation}
En plus de la diffusion, considérons des modèles où le taux d'accroissement de $u$ depend aussi de la densité $u$.\\
Ceci donne les équations de reaction-diffusion:
\begin{equation}\dt u = \Delta u + F(u) \label{eq:ReaDi} \end{equation} 
où F est assez lisse.\\
Il est souvent naturel dans les modêles de considérer $F(u)$ proportionel à  $u$ pour $u$ petit (``croissance"), et quand $u$ devient proche de 1, l'accroissement $F(u)$ s'arrête: $F(1)=0$ (``saturation").\\
Ces types de modèles ont étés introduits et examinés par les travaux de Fisher[1] et Kolmogorov, Petrovsky et Piscounuv[2] (abrégés KPP).\\
Un exemple d'une telle équation est:

\begin{equation}
	\dt u = \Delta u +u(1-u) \label{eq:KPP}
\end{equation}
qui sera dans la suite étudiée dans le cas 1-dimensionel en $x$ : $u=u(x,t)$.

\subsection{Réaction}
En observant les solutions constantes en $x$: $u(x,t)=v(t)$ dans \eqref{eq:KPP}, l'équation différentielle ordinaire (EDO ou ODE): \begin{equation}
	\dt v = v - v^2 = F(v)
\end{equation}
est obtenue. \\
Il y a deux équilibres ($F(v)=0)$) pour $v=0$ et $v=1$.Par le théorême de stabilité de Lyapunov, $F'(0)>0$ montre que $v=0$ est instable et $F'(1)<0$ montre $v=1$ is asymptotiquement stable.

\subsection{Réaction-Diffusion}
Dans l'éspace $X=C^0_{b,unif}(\mathbb{R},\mathbb{R})$ des fonctions bornées et uniformément continues, il y a existence locale et unicité des solutions de l'éqaution de Fisher-KPP \eqref{eq:ReaDi}. Grâce à un principe du maximum, il y a aussi existence globale et unicité des solutions.
\newtheorem{theorem}{Théorème}
\begin{theorem}:\\ Existence et Unicité de la solution dans $X$: Soit $U_0 \in X$. Il existe une unique solution de l'équation de Fisher-KPP \eqref{eq:ReaDi}  $U \in C([0,\infty[,X)$ avec condition initiale $U_0$. \end{theorem}
\begin{theorem}:\\ Principe du Maximum: Soit $u_1$ et $u_2$ deux solutions de \eqref{eq:ReaDi}. Si il éxiste $t_0$ tel que $u_1(x,t_0)<u_2(x,t_0) $ $\forall x$ alors $u_1(x,t)<u_2(x,t)$  $ \forall x$ et $\forall t>t_0$ \end{theorem}

\subsection{Solutions en onde plane stationaire / onde progressive}
Rappellons la définition d'une solution en onde plane stationaire / onde progressive:
\begin{definition}{Solutions en onde plane stationaire}\\Une solution en onde plane stationnaire est une solution de la forme $u(x,t)=h(x-st)$ où $c \in \mathbb{R} $.\\ On fera parfois l'abus de notation $u(x,t)=u(x-st)$
\end{definition}
Sous des hypotheses ``faibles" sur $F$, l'équation \eqref{eq:ReaDi}: $\dt u = \Delta u + F(u)$ a alors la propriété surprenante et importante de posséder des solutions en ondes planes stationaires liant les états d'équilibre $u=1$ (à $-\infty$) et $u=0$ (à $+\infty$).\\
Les hypothèses sur $F$ portent en partie sur le fait que \eqref{eq:ReaDi} doit posséder:\\
- Deux états d'équilibre $u=1$ et $u=0$: $F(0)=F(1)=0$:\\
- Un phénomêne de ``croissance" : $F'(0)>0$\\
- Un phénomêne de ``saturation" : $F'(1)<0$ \\
\paragraph{Étude des solutions en ondes progressive de \eqref{eq:ReaDi}}:\\
En substituant $u(x,t) = h(x-st) = h(y)$ pour $y=x-st$ dans \eqref{eq:ReaDi}, les équations obtenues sur $h$ sont: \begin{equation} \label{eq:FWS} \left\{
                \begin{array}{ll}
                h''(y)+ sh'(y)+F(h(y))=0 \\
                h(-\infty)= 1 \\  h(+\infty) =0 
                   \end{array}
              \right.
\end{equation} 
qui est une équation élliptique non linéaire. Le problème est donc de trouver $s$ et $h \in C^2$ tels que le systême \eqref{eq:FWS} soit vérifié. Le théorème obtenu est le suivant:



\begin{theorem}:\\
Soit $F \in C^1([0,1])$ tel $F(0)=F(1)=0$ et $F\geq 0$. \\
Il existe une vitesse critique $s_*$ telle que $s_*^2 \geq 4F'(0)$ et: \\ \\
- i) $\forall s \geq s_*$, l'équation \eqref{eq:FWS} a une solution $h_s:\mathbb{R} \rightarrow ]0,1[$ de classe $C^3$.\\ Cette solution est unique à translation près. \\
- ii)  $\forall c<c_*$ l'équation \eqref{eq:FWS} n'a pas de solution $h:\mathbb{R} \rightarrow [0,1]$
\end{theorem}
Remarque : Dans le second cas il existe des solutions mais elles ne sont pas confinées dans [0,1] ni dans $\mathbb{R}^+$, ce qui ne fait pas de sens dans une étude de densité de population.

\newpage
\section{Dyamique de Réseaux en Croissance}
Dans cette séction et la suite nous étudions le modèle sur la croissance de réseaux dynamiques branchants, par éxemple un champignon, proposé par Rémi Catellier, Yves D'Angelo et Cristiano Ricci, avec rescaling adéquat:
\begin{equation}\label{eq:BDNG}  \left\{
                \begin{array}{ll}
                \dt\mu + \nabla(\mu v) = f(C)(\mu + \rho) -\mu\rho \\
                   \dt(\mu v)+\nabla(\mu v\times v) +T\nabla\mu=-\lambda\mu v+\mu\nabla C-\mu v \rho \\
                 \dt\rho=  F(v) \mu \\
                  \dt C = -b\rho C
                \end{array}
              \right.
\end{equation} 
The unknown $\mu$ represents the density of the apices of the fungus.\\ The unknown $\rho$ represents the density of the network.\\
The unknown $v$ represents the speed of the apices.\\
The unknown $C$ represents the concentration of nutrient.\\
The parameters $T$, $\lambda$ and $b$ are scalar constants that represent temperature, fluid damping on the speed of the apexes, and the rate of consumption of the nutrients by the network.\\
The function $f$ indicates the influence of the concentration of nutrient on the growth of the fungus. Usualy, to have a stationary state on the growth of the fungus, we need $f(0)=0$ and $f(x)/x$ in $L^1$ near 0.\\
The function $F$ represents the inverse of the average time spent by apexes in a given point, and is given by the expression:
\begin{equation}
	F(V)=(\frac{1}{2\pi T})^\frac{d}{2}\int_{\mathbb{R}^d} |v|\exp(-\frac{|v-V|^2}{2T})dv\end{equation}
where d is the dimension of the problem. This model is often simplified by substituting $F(V)$ with a constant: $F(V)= F_0$ .\\

\subsection{Explanation of the terms in equation \eqref{eq:BDNG}}
The fungus is a branching dynamical network that can be studied in two parts: the apexes (tips of the newtwork) and the hyphen (branches of the network).\\
Looking at each line of equation \eqref{eq:BDNG} seperately, the model has:\\
i) The first line is the mass balance equation on the apexes with classical left term $ \dt\mu + \nabla(\mu v)  $. The right term is composed of $f(C)(\mu + \rho)$ corresponding to a growth of the number of apexes depending on the concentration of nutrient and the existing mass of apexes and hyphen, and a term $-\mu\rho$ which corresponds to anastomosis : a tip that encounters a branch will merge with it and be destroyed. There is a growth term and a saturation term like the KPP model.
\\ ii)  The second line is the momentum balance equation of the apexes with classical left term $ \dt(\mu v)+\nabla(\mu v\times v) $. The term $T\nabla\mu$ represents a brownian motion followed by the apexes. The term $-\lambda\mu v$ represents a fluid damping in the physics of the problem. The term $+\mu\nabla C$ represents of proponency of the apexes to go where the nutrient concentation is dense. The term $-\mu v \rho $ represents the loss of momentum due to anastomosis.\\
iii) The third line describes the relationship between apexes and hyphen: the trail of the apexes are the branches.\\
iv) The fourth line describes the evolution of the nutrient concentration: it is eaten by the hyphen with rate $bC$. 
%\end{document}