
\ifdefined\COMPLETE
\else
\documentclass[11pt]{article}
\usepackage[french, english]{babel}
\usepackage[utf8]{inputenc}
\usepackage{graphicx}
\usepackage{framed}
\usepackage[normalem]{ulem}
\usepackage{amsmath}
\usepackage{amsthm}
\usepackage{amssymb}
\usepackage{amsfonts}
\usepackage{enumerate}
\usepackage{import}
\usepackage[top=1 in,bottom=1in, left=1 in, right=1 in]{geometry}
\usepackage{listingsutf8}
\usepackage{color}
\usepackage{float}
\usepackage{graphicx}
\usepackage{subcaption}
\usepackage[toc,page]{appendix}
\usepackage{multicol}
\usepackage{wrapfig}
\usepackage{sidecap}

\floatstyle{boxed} 
\restylefloat{figure}
\definecolor{mygreen}{rgb}{0,0.6,0}
\definecolor{mygray}{rgb}{0.5,0.5,0.5}
\definecolor{mymauve}{rgb}{0.58,0,0.82}
\newcommand{\dt}{\partial_t}
\newcommand{\Tl}{\frac{T}{\lambda}}
\theoremstyle{definition}
\newtheorem{definition}{Définition}[section]
\DeclareMathOperator*{\argmax}{arg\,max}
\DeclareMathOperator*{\argmin}{arg\,min}
 


\lstset{ 
  backgroundcolor=\color{white},   % choose the background color; you must add \usepackage{color} or \usepackage{xcolor}; should come as last argument
  basicstyle=\footnotesize,        % the size of the fonts that are used for the code
  breakatwhitespace=false,         % sets if automatic breaks should only happen at whitespace
  breaklines=true,                 % sets automatic line breaking
  captionpos=b,                    % sets the caption-position to bottom
  commentstyle=\color{mygreen},    % comment style
  deletekeywords={...},            % if you want to delete keywords from the given language
  escapeinside={\%*}{*)},          % if you want to add LaTeX within your code
  extendedchars=true,              % lets you use non-ASCII characters; for 8-bits encodings only, does not work with UTF-8
  firstnumber=1000,                % start line enumeration with line 1000
  frame=single,	                   % adds a frame around the code
  keepspaces=true,                 % keeps spaces in text, useful for keeping indentation of code (possibly needs columns=flexible)
  keywordstyle=\color{blue},       % keyword style
  language=Python,                 % the language of the code
  morekeywords={*,...},            % if you want to add more keywords to the set
  numbers=left,                    % where to put the line-numbers; possible values are (none, left, right)
  numbersep=5pt,                   % how far the line-numbers are from the code
  numberstyle=\tiny\color{mygray}, % the style that is used for the line-numbers
  rulecolor=\color{black},         % if not set, the frame-color may be changed on line-breaks within not-black text (e.g. comments (green here))
  showspaces=false,                % show spaces everywhere adding particular underscores; it overrides 'showstringspaces'
  showstringspaces=false,          % underline spaces within strings only
  showtabs=false,                  % show tabs within strings adding particular underscores
  stepnumber=2,                    % the step between two line-numbers. If it's 1, each line will be numbered
  stringstyle=\color{mymauve},     % string literal style
  tabsize=2,	                   % sets default tabsize to 2 spaces
  title=\lstname                   % show the filename of files included with \lstinputlisting; also try caption instead of title
}
\lstset{inputencoding=utf8/latin1}
\newcommand{\Dt}{\Delta t}
\newcommand{\Dx}{\Delta x}
 %file containing all the used libraries
\begin{document}
\fi

\section{L'équation de Fisher ou KPP}
\subsection{Préliminaire}
Notre point de départ est l'équation de diffusion:\begin{equation}\dt u = \Delta u  \end{equation}
En plus de la diffusion, considérons des modèles où le taux d'accroissement de $u$ dépend aussi de la densité $u$.\\
Ceci donne les équations de réaction-diffusion:
\begin{equation}\dt u = \Delta u + F(u) \label{eq:ReaDi} \end{equation} 
où $F$ est assez lisse.\\
Il est souvent naturel dans les modèles de considérer $F(u)$ proportionnel à  $u$ pour $u$ petit (``croissance"), et quand $u$ devient proche de 1, l'accroissement $F(u)$ s'arrête: $F(1)=0$ (``saturation").\\
Ces types de modèles ont étés introduits et examinés par les travaux de Fisher[1] %CITER FISHER
et Kolmogorov, Petrovsky et Piscounuv (abrégés KPP).\\ %CITER KPP
Un exemple d'une telle équation est:

\begin{equation}
	\dt u = \Delta u + ru(1-u) \label{eq:KPP}
\end{equation}
où $r>0$, qui sera dans la suite étudiée dans le cas 1-dimensionnel en $x$ : $u=u(x,t)$.

\subsection{Réaction}
En observant les solutions constantes en $x$: $u(x,t)=v(t)$ dans \eqref{eq:KPP}, l'équation différentielle ordinaire (EDO ou ODE) suivante est obtenue: \begin{equation}
	\dt v = r(v - v^2) = F(v)
\end{equation}
Il y a deux équilibres ($F(v)=0)$) pour $v=0$ et $v=1$.\\
Par le théorème de stabilité de Lyapunov, $F'(0)>0$ montre que $v=0$ est instable et $F'(1)<0$ montre $v=1$ est asymptotiquement stable.

\subsection{Réaction-Diffusion}
Dans l’espace $X=C^0_{b,unif}(\mathbb{R},\mathbb{R})$ des fonctions bornées et uniformément continues, il y a existence locale et unicité des solutions de l’équation de Fisher-KPP \eqref{eq:ReaDi}. Grâce à un principe du maximum, il y a aussi existence globale et unicité des solutions.
\newtheorem{theorem}{Théorème}
\begin{theorem} \textbf{Existence et Unicité de la solution de Fisher-KPP dans $X$:}\\ Soit $U_0 \in X$. Il existe une unique solution de l'équation de Fisher-KPP \eqref{eq:ReaDi}  $U \in C([0,\infty[,X)$ avec condition initiale $U_0$. \end{theorem}
\begin{theorem} \textbf{Principe de comparaison:}
\\Soient $u_1$ une sous-solution $C^{2,1}$ et $u_2$ une sur-solution $C^{2,1}$ de  \eqref{eq:ReaDi}.\\ Si  $u_1(x,0)< u_2(x,0)$  $\forall x$ alors $u_1(x,t)<u_2(x,t)$  $ \forall x$ et $\forall t>0$.\\
En particulier si $u$ est une solution de \eqref{eq:ReaDi} telle que $u_1(x,0)< u(x,0)< u_2(x,0) $  $\forall x$  alors: \begin{equation}
 u_1(x,t)< u(x,t) < u_2(x,t) \  \forall x, \forall t>0
\end{equation}
\end{theorem}
\newpage
\begin{proof} Supposons par l'absurde qu'il existe $ t_0 = \inf\{ t >0, \exists x / u_2(x,t)= u_1(x,t) \} $
. \\
Soit $x_0= \inf \{x/ u_1(x,t_0)=u_2(x,t_0) \} $ Par définition de $x_0$ et $t_0$ et comme $u_1$ et $u_2$ sont $C^1$ par rapport à t et comme $u_2(x_0,0)<u_1(x_0,0)$, il existe un voisinage de $t_0$ tel que $\partial_t (u_2-u_1)(x_0,.) <0$. \\Comme de plus $u_1$ est une sur-solution et $u_2$ une sous solution, et $F(u_1(x_0,t_0))=F(u_2(x_0,t_0))$  on en déduit:
\begin{equation}
	\partial_{xx}^2(u_1-u_2)(x_0,t_0) < 0 .
\end{equation}
Mais $x_0$ est un minimum local de la fonction $x\mapsto (u_1-u_2)(t_0,x)$, ce qui contredit l'inégalité précédente donc l'existence de $t_0$.
\end{proof}
\subsection{Solutions d'ondes plane stationnaire / onde progressive}
Rappelons la définition d'une solution en onde plane stationnaire / onde progressive:
\begin{definition}{\textbf{Solutions en onde plane stationnaires.}}\\Une solution en onde plane stationnaire est une solution de la forme $u(x,t)=h(x-st)$ où $c \in \mathbb{R}$.\\ On fera parfois l'abus de notation $u(x,t)=u(x-st)$
\end{definition}
Sous des hypothèses ``faibles" sur $F$, l'équation \eqref{eq:ReaDi}: $\dt u = \Delta u + F(u)$ a alors la propriété surprenante et importante de posséder des solutions en ondes planes stationnaires liant les états d'équilibre $u=1$ (à $-\infty$) et $u=0$ (à $+\infty$).\\
Les hypothèses sur $F$ portent en partie sur le fait que \eqref{eq:ReaDi} doit posséder:\\
- Deux états d'équilibre $u=1$ et $u=0$: $F(0)=F(1)=0$:\\
- Un phénomène de ``croissance" : $F'(0)>0$\\
- Un phénomène de ``saturation" : $F'(1)<0$ \\
\paragraph{Étude des solutions en ondes progressive de \eqref{eq:ReaDi}}:\\
En substituant $u(x,t) = h(x-st) = h(y)$ pour $y=x-st$ dans \eqref{eq:ReaDi}, les équations obtenues sur $h$ sont: \begin{equation} \label{eq:FWS} \left\{
                \begin{array}{ll}
                h''(y)+ sh'(y)+F(h(y))=0 \\
                h(-\infty)= 1 \\  h(+\infty) =0 
                   \end{array}
              \right.
\end{equation} 
qui est une équation elliptique non linéaire. Le problème est donc de trouver $s$ et $h \in C^2$ tels que le système \eqref{eq:FWS} soit vérifié. Le théorème obtenu est le suivant:



\begin{theorem}{\textbf{Existence de solutions en onde progressive pour les équations de réaction-diffusion}}:\\
Soit $F \in C^1([0,1])$ tel $F(0)=F(1)=0$ et $F\geq 0$. \\
Il existe une vitesse critique $s_*$ telle que $s_*^2 \geq 4F'(0)$ et: \\ \\
- i) $\forall s \geq s_*$, l'équation \eqref{eq:FWS} a une solution $h_s:\mathbb{R} \rightarrow ]0,1[$ de classe $C^3$.\\ Cette solution est unique à translation près. \\
- ii)  $\forall s<s_*$ l'équation \eqref{eq:FWS} n'a pas de solution $h:\mathbb{R} \rightarrow [0,1]$
\end{theorem}
\textbf{Remarques}:\\
 Dans le cas ii) il existe des solutions en ondes planes mais elles ne sont pas confinées dans [0,1] ni dans $\mathbb{R}^+$, ce qui ne fait pas de sens dans une étude de densité de population.\\
 Dans le cas de l'équation de Fisher-KPP, c'est à dire pour $F(u)=r(u - u^2)$, on a $s_*^2 = 4F'(0) = 4r $: la vitesse minimale de propagation est $s^* =2\sqrt r $.

\newpage
\subsection{Dans l'exemple de Fisher-KPP}
Considérons l'équation de Fisher-KPP \eqref{eq:KPP}: $\dt u = \Delta u + ru(1-u)$.\\
Comme $u \equiv 0$ et $u \equiv 1$ sont des solutions particulières de \eqref{eq:KPP}, si $0\leq u_0(x) \leq 1 \ \forall x$, alors par le principe du maximum on a $0\leq u(x,t) \leq 1 \ \forall x,t $.\\
Soit $h$ une solution en onde plane de \eqref{eq:FWS} avec $0\leq h \leq 1 \ \forall y$, i.e. $h''(y)+ sh'(y)+rh(y)-rh^2(y)=0$. 
En linéarisant autour de l'état $h=0$ on obtient: \begin{equation}
 h''(y)+ sh'(y)+rh(y)=0
\end{equation}
de polynôme caractéristique $X^2+sX+r=0$ et de discriminant $\Delta = s^2 - 4r $.\\
On voit alors que la condition $s^2 \geq 4r$ est nécessaire pour que $0\leq h \leq 1 $: c'est la condition d'amortissement fort de l’oscillateur autour de l’état $h=0$.


\subsection{Théorèmes de sélection de la vitesse pour KPP}
Le théorème important suivant est du aux travaux de Kolmogorov, Petrovsky et Piscounuv de 1937. C'est l'article et le résultat fondateur de la théorie des ondes planes dans les systèmes de réaction-diffusion. %CITER KPP\\

\begin{theorem}{\textbf{Convergence vers une solution d'onde à vitesse minimale pour les solutions de l'équation de Fisher-KPP avec une donnée initiale à support compact }}\\
Soit $u_0 \to ]0,1[$ une donnée initiale à support compact. 
Soit $u$ la solution de l'équation de Fisher-KPP \eqref{eq:KPP} avec $r=1$ et de donnée initiale $u_0$.
Alors quand $t \to \infty$, $u$ converge uniformément en $x$ vers une solution d'onde $h_{s^*}$ de \eqref{eq:FWS} qui se de déplace à vitesse minimale $s^* =2$ :
\begin{equation*}
\sup_{y \in \mathbb{R}}  |u(y+m(t),t)-h_{s^*}(y)| \to_{t\to \infty} 0
\end{equation*}
où $m(t)= 2t - (3/2)\log (t) + y_0 $.
\end{theorem}
Remarque: La vitesse du front est alors $s(t) = \dt m(t) = 2 - \frac{3}{2t} \to_{t\to \infty} 2 $.
\paragraph{}
Ce résultat a été raffiné par la suite par Uchiyama, Bramson et Lau. Leurs travaux apportent plus d'informations sur comment la vitesse du front se sélectionne en fonction de la donnée initiale, et comment il est possible d'obtenir d'autres vitesses de fronts que la vitesse minimale en fonction de la donnée initiale.
\begin{theorem}{\textbf{Sélection de la vitesse pour les solutions de l'équation de Fisher-KPP en fonction de la donnée initiale}}\\
Si $u_0 \to ]0,1[$ vérifie $\lim\inf_{x\to -\infty} u_0(x) > 0$ et $\int_0^{+\infty} xe^xu_0(x) / dx < \infty$\\
alors il existe $y_0 \in \mathbb{R}$ tel que la solution de \eqref{eq:KPP} avec données initiales $u_0$ vérifie
\begin{equation*}
\sup_{y \in \mathbb{R}}  |u(y+m(t),t)-h_{s^*}(y)| \to_{t\to \infty} 0
\end{equation*}
où $m(t)= 2t - (3/2)\log (t) + y_0 $. 
\end{theorem}
D'autres vitesses peuvent être sélectionnées: Si la donnée initiale vérifie $u_0(x) \approx  e^{-\lambda_-(s)x}$ quand $x \to +\infty$, où $\lambda_-(s)$ est la plus petite racine du polynôme caractéristique $X^2+sX+r=0$, alors la solution converge vers une onde progressive de vitesse $s$.

\ifdefined\COMPLETE
\else
\end{document}
\fi