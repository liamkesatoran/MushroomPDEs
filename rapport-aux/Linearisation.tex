\ifdefined\COMPLETE
\else
\documentclass[11pt]{article}
\usepackage[french, english]{babel}
\usepackage[utf8]{inputenc}
\usepackage{graphicx}
\usepackage{framed}
\usepackage[normalem]{ulem}
\usepackage{amsmath}
\usepackage{amsthm}
\usepackage{amssymb}
\usepackage{amsfonts}
\usepackage{enumerate}
\usepackage{import}
\usepackage[top=1 in,bottom=1in, left=1 in, right=1 in]{geometry}
\usepackage{listingsutf8}
\usepackage{color}
\usepackage{float}
\usepackage{graphicx}
\usepackage{subcaption}
\usepackage[toc,page]{appendix}
\usepackage{multicol}
\usepackage{wrapfig}
\usepackage{sidecap}

\floatstyle{boxed} 
\restylefloat{figure}
\definecolor{mygreen}{rgb}{0,0.6,0}
\definecolor{mygray}{rgb}{0.5,0.5,0.5}
\definecolor{mymauve}{rgb}{0.58,0,0.82}
\newcommand{\dt}{\partial_t}
\newcommand{\Tl}{\frac{T}{\lambda}}
\theoremstyle{definition}
\newtheorem{definition}{Définition}[section]
\DeclareMathOperator*{\argmax}{arg\,max}
\DeclareMathOperator*{\argmin}{arg\,min}
 


\lstset{ 
  backgroundcolor=\color{white},   % choose the background color; you must add \usepackage{color} or \usepackage{xcolor}; should come as last argument
  basicstyle=\footnotesize,        % the size of the fonts that are used for the code
  breakatwhitespace=false,         % sets if automatic breaks should only happen at whitespace
  breaklines=true,                 % sets automatic line breaking
  captionpos=b,                    % sets the caption-position to bottom
  commentstyle=\color{mygreen},    % comment style
  deletekeywords={...},            % if you want to delete keywords from the given language
  escapeinside={\%*}{*)},          % if you want to add LaTeX within your code
  extendedchars=true,              % lets you use non-ASCII characters; for 8-bits encodings only, does not work with UTF-8
  firstnumber=1000,                % start line enumeration with line 1000
  frame=single,	                   % adds a frame around the code
  keepspaces=true,                 % keeps spaces in text, useful for keeping indentation of code (possibly needs columns=flexible)
  keywordstyle=\color{blue},       % keyword style
  language=Python,                 % the language of the code
  morekeywords={*,...},            % if you want to add more keywords to the set
  numbers=left,                    % where to put the line-numbers; possible values are (none, left, right)
  numbersep=5pt,                   % how far the line-numbers are from the code
  numberstyle=\tiny\color{mygray}, % the style that is used for the line-numbers
  rulecolor=\color{black},         % if not set, the frame-color may be changed on line-breaks within not-black text (e.g. comments (green here))
  showspaces=false,                % show spaces everywhere adding particular underscores; it overrides 'showstringspaces'
  showstringspaces=false,          % underline spaces within strings only
  showtabs=false,                  % show tabs within strings adding particular underscores
  stepnumber=2,                    % the step between two line-numbers. If it's 1, each line will be numbered
  stringstyle=\color{mymauve},     % string literal style
  tabsize=2,	                   % sets default tabsize to 2 spaces
  title=\lstname                   % show the filename of files included with \lstinputlisting; also try caption instead of title
}
\lstset{inputencoding=utf8/latin1}
\newcommand{\Dt}{\Delta t}
\newcommand{\Dx}{\Delta x}
 %file containing all the used libraries
\begin{document}
\fi

\section{Recherche de la vitesse d'onde des solutions progressives de l’Équation KPP avec Mémoire}
On a le modèle suivant: 
\begin{equation} \label{eq:FWSMem} \left\{
                \begin{array}{ll}
                   \dt\mu -K\Delta\mu = f(C)(\mu + \rho) -\mu\rho\\
                 \dt\rho=  F_0 \mu \\
                  \dt C = -b\rho C
                \end{array}
              \right.
\end{equation} 
où $f(0)=0$ et $f$ est positive. Typiquement, $f(C)=C$:\\
On recherche des solutions en onde plane, on pose $s$ la vitesse d'onde et $\xi = x - st$. \\
Par abus de notation, on pose $\mu(\xi)=\mu(x,t)$, $\rho(\xi)=\rho(x,t)$, etc... \\
On a alors:
\begin{equation} \left\{ \begin{array}{ll} -s \mu'-K\mu''=f(C)(\mu+\rho)-\mu\rho \\ -s\rho' = F_0\mu  \\C'=\frac{b\rho C}{s} \end{array}\right.
\end{equation}
Nos états stationnaires (i.e. qui correspondent à des derivées nulles) sont: \begin{equation}
	(\mu,\rho,C) = \left\{ \begin{array}{ll} (0,0,C_0) \\
 (0,\rho_\infty,0) , \rho_\infty > 0 \end{array} \right.
\end{equation}
\subsection{Linéarisation au voisinage de $(0,0,C_0)$}
Au voisinage de $(0,0,C_0)$ on a, en posant $f(C_0)=f_0$, la linéarisation de \ref{eq:FWSMem} :
\begin{equation} \left\{ \begin{array}{ll} -s \mu'-K\mu''=f_0(\mu+\rho) \\ -s\rho' = F_0\mu   \end{array}\right.
\end{equation} ce qui devient:  \begin{equation} \rho''' +\frac{s}{K}\rho''+\frac{f_0}{K}\rho'-\frac{F_0f_0}{Ks}\rho =0 \end{equation} de polynôme caractéristique: \begin{equation} P(X)= X^3 +\frac{s}{K}X^2+\frac{f_0}{K}X-\frac{F_0f_0}{Ks}.\end{equation}
Le signe de $s$ correspondant à la direction de propagation, il y'a symétrie en $s$: on prend içi $s<0$ ce qui correspond a une propagation vers la gauche.\\
Pour $s<0$, $P$ est de degŕe 3 et $P(0)>0$ donc $P$ a une racine négative $r_1$ .\\
Pour conserver la positivité autour de l'état $(0,0,C_0)$ il faut que les racines de $P$ soit réelles: sinon on obtient osillations autour de 0.\\ 
Pour que $P$ ait deux autres racines réelles $r_3>r_2>r_1$ il faut (condition nécessaire et suffisante) que $P'$ s'annule deux fois et que le discriminant $\Delta$ de $P$ soit positif.
\subsubsection{Première condition: $P'$ a deux annulations:}
$P'(X)= 3X^2+ 2\frac{s}{K}X+ \frac{f_0}{K}$ a pour discriminant $\Delta'=4\frac{1}{K^2}(s^2-3Kf_0)$ ce qui donne la condition \begin{equation} \label{eq:condition_P'}
	\boxed{s^2 >3K f_0.
	}
\end{equation}
\subsubsection{Deuxième condition: $\Delta>0$:}
Pour $P=aX^3 +bX^2 + cX + d$ on a $\Delta= b^2c^2 +18abcd-27a^2d^2 -4ac^3 -4b^3d$ ce qui dans notre cas donne
\begin{align*}
	\Delta=\frac{1}{K^4}f_0^2s^2 -18 \frac{f_0^2F_0}{K^3}-27 \frac{F_0^2 f_0^2}{K^2s^2} - 4 \frac{f_0^3}{K^3}+4 \frac{F_0f_0s^2}{K^4} \\ = s^2 \frac{f_0(f_0+4F_0)}{K^4}- \frac{f_0^2(18F_0+4)}{K^3} -\frac{27F_0^2f_0^2}{K^2} \frac{1}{s^2}\\ 
=	\frac{f_0}{K^4s^2}[(f_0+4F_0)s^4-Kf_0(18F_0+4) s^2 - 27 K^2F_0^2f_0]. \end{align*}
On est revenu à étudier le signe du polynôme en $s^2$: \begin{equation}
	D(s^2)=(f_0+4F_0)s^4-Kf_0(18F_0+4)s^2 - 27 K^2F_0^2f_0
\end{equation} 
de discriminant $d$:
\begin{align*}
	d=\Big(Kf_0(18F_0+4) \Big)^2 +108(f_0+4F_0)K^2F_0^2f_0 \\ 
	= K^2f_0(f_0(18F_0+4)^2+108(f_0+4F_0)F_0^2) >0.
\end{align*}
On obtient donc la condition sur la positivité de $\Delta$: 
\begin{equation}\boxed{
	s^2> K\frac{f_0(18F_0+4)+\sqrt{f_0(f_0(18F_0+4)^2+108(f_0+4F_0)F_0^2)}}{2(f_0+4F_0)}.
	}\label{eq:condition_Delta}
\end{equation}
\subsubsection{Signe des racines au voisinage de $(0,0,C_0)$}
On sait déjà que $r_3<0$. Comme $r_1r_2r_3<0$, on remarque que $r_2$ et $r_1$ sont du même signe.\\
De plus $P'$ a un axe de symétrie $X=-\frac{s}{3K}>0$ car $s<0$ donc $P$ atteint un minimum local (forcement négatif) en un point positif donc $P$ a une racine positive.\\
On en déduit $r_1>r_2>0$: \\ 
Sous les conditions \eqref{eq:condition_P'} et \eqref{eq:condition_Delta}, $P$ a deux racines positives et une négative.

\paragraph{Conclusion}
Comme pour l'équation de KPP, la linéarisation autour de l'étât $(0,0,C_0)$ fait apparaitre une condition sur $s$ nécessaire pour preserver la positivité.
\subsection{Linéarisation au voisinage de $(0,\rho_\infty,0)$} 
Autour de $(0,\rho_\infty,0)$:
Posons $(\mu,\rho,C)=(\mu, \rho_\infty + \epsilon, C)$. On a \\
\begin{equation} \left\{ \begin{array}{ll} -s \mu'-K\mu''=f(C)\rho_\infty-\mu\rho_\infty\\C'=\frac{b\rho_\infty C}{s} \\
-s\epsilon'= F_0\mu. \end{array}\right.
\end{equation}
La deuxième ligne donne \begin{equation}
C(y) = \Lambda\exp(\frac{b\rho_\infty}{s}y )
\end{equation}
et la réunion de la première et la troisième se traduit sur $\epsilon$ par:
\begin{equation}
	s^2 \epsilon''+Ks\epsilon'''=f(C)F_0\rho_\infty+s\epsilon'\rho_\infty
\end{equation} qui est une EDO d'ordre trois en $\epsilon$ avec terme source $\frac{F_0f(C)}{Ks} \rho_\infty$ de polynôme caracteristique: \begin{equation}
	Q(X)=X^3+\frac{s}{K}X^2-\frac{\rho_\infty}{K}X
\end{equation}
qui possède toujours trois racines:  $0$, une négative et une positive: $X= - \frac{1}{2K}(s \pm \sqrt{s^2+4\rho_\infty Ks}).$
Sur $\mu$ on a: \begin{equation}
	-s \mu'-Ks\mu''=f(C)\rho_\infty-\mu\rho_\infty.
\end{equation}
Dans le cas $f(C)=C$:\\
$\mu$ a pour polynôme caractéristique homogène $M(X)=X^2 +\frac{1}{K}X - \frac{\rho_\infty}{Ks}$ de racines:\\$r_{+,-}= - \frac{1}{2K}(1 \pm \sqrt{1+4 \frac{\rho_\infty K}{s}})$ \\ donc $\mu_H = Ae^{r_+y}+ Be^{r_-y}$ (On choisit $r_+>0, r_-<0$).\\
En cherchant une solution particulière de la forme $\mu_p =M\exp(\frac{b\rho_\infty}{s}y )$ on obtient $M = - \frac{\Lambda}{b^2\rho_\infty K + b - 1}$ et donc $\mu=Ae^{r_+y}+ Be^{r_-y}+M e^{\frac{b\rho_\infty}{s}y }$
et donc $\rho = \rho_\infty + \alpha e^{r_+y} + \beta e^{r_-y} + \frac{Ms}{b\rho_\infty}\exp(\frac{b\rho_\infty}{s}y)$.
\paragraph{Conclusion}
Comme pour l'équation de KPP, on obtient à priori pas de condition sur $s$ suite à la linéarisation autour de l'étât $(0,\rho_\infty,0)$ mais seulement des informations sur la dynamique autour de ces états.
\ifdefined\COMPLETE
\else
\end{document}
\fi