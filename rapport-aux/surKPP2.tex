
\ifdefined\COMPLETE
\else
\documentclass[11pt]{article}
\usepackage[french, english]{babel}
\usepackage[utf8]{inputenc}
\usepackage{graphicx}
\usepackage{framed}
\usepackage[normalem]{ulem}
\usepackage{amsmath}
\usepackage{amsthm}
\usepackage{amssymb}
\usepackage{amsfonts}
\usepackage{enumerate}
\usepackage{import}
\usepackage[top=1 in,bottom=1in, left=1 in, right=1 in]{geometry}
\usepackage{listingsutf8}
\usepackage{color}
\usepackage{float}
\usepackage{graphicx}
\usepackage{subcaption}
\usepackage[toc,page]{appendix}
\usepackage{multicol}
\usepackage{wrapfig}
\usepackage{sidecap}

\floatstyle{boxed} 
\restylefloat{figure}
\definecolor{mygreen}{rgb}{0,0.6,0}
\definecolor{mygray}{rgb}{0.5,0.5,0.5}
\definecolor{mymauve}{rgb}{0.58,0,0.82}
\newcommand{\dt}{\partial_t}
\newcommand{\Tl}{\frac{T}{\lambda}}
\theoremstyle{definition}
\newtheorem{definition}{Définition}[section]
\DeclareMathOperator*{\argmax}{arg\,max}
\DeclareMathOperator*{\argmin}{arg\,min}
 


\lstset{ 
  backgroundcolor=\color{white},   % choose the background color; you must add \usepackage{color} or \usepackage{xcolor}; should come as last argument
  basicstyle=\footnotesize,        % the size of the fonts that are used for the code
  breakatwhitespace=false,         % sets if automatic breaks should only happen at whitespace
  breaklines=true,                 % sets automatic line breaking
  captionpos=b,                    % sets the caption-position to bottom
  commentstyle=\color{mygreen},    % comment style
  deletekeywords={...},            % if you want to delete keywords from the given language
  escapeinside={\%*}{*)},          % if you want to add LaTeX within your code
  extendedchars=true,              % lets you use non-ASCII characters; for 8-bits encodings only, does not work with UTF-8
  firstnumber=1000,                % start line enumeration with line 1000
  frame=single,	                   % adds a frame around the code
  keepspaces=true,                 % keeps spaces in text, useful for keeping indentation of code (possibly needs columns=flexible)
  keywordstyle=\color{blue},       % keyword style
  language=Python,                 % the language of the code
  morekeywords={*,...},            % if you want to add more keywords to the set
  numbers=left,                    % where to put the line-numbers; possible values are (none, left, right)
  numbersep=5pt,                   % how far the line-numbers are from the code
  numberstyle=\tiny\color{mygray}, % the style that is used for the line-numbers
  rulecolor=\color{black},         % if not set, the frame-color may be changed on line-breaks within not-black text (e.g. comments (green here))
  showspaces=false,                % show spaces everywhere adding particular underscores; it overrides 'showstringspaces'
  showstringspaces=false,          % underline spaces within strings only
  showtabs=false,                  % show tabs within strings adding particular underscores
  stepnumber=2,                    % the step between two line-numbers. If it's 1, each line will be numbered
  stringstyle=\color{mymauve},     % string literal style
  tabsize=2,	                   % sets default tabsize to 2 spaces
  title=\lstname                   % show the filename of files included with \lstinputlisting; also try caption instead of title
}
\lstset{inputencoding=utf8/latin1}
\newcommand{\Dt}{\Delta t}
\newcommand{\Dx}{\Delta x}
 %file containing all the used libraries
\newtheorem{theorem}{Théorème}
\begin{document}
\fi

La preuve de ces résultats ne sera pas étudiée dans ce rapport mais nous allons montrer un résultat un peu plus faible pour une condition initiale  $\mathbf{1}_{x<0}$ et pour la fonction de KPP classique $F(u) = u(1-u)$: on va montrer que sa vitesse limite "faible" est égale à 2.

\begin{definition}\textbf{Vitesse limite faible de propagation}\\
$u(x,t)$ se propage faiblement à la vitesse $s^*$ si et seulement si: \[  \begin{cases} \
\inf\limits_{x\leq st}{u(x,t) \to 1 } \text{ si $s<s^*$}\\ \ \sup\limits_{x\geq st}{u(x,t) \to 0 } \text{ si $s>s^*$}
\end{cases} \]
\end{definition}
Pour montrer que la vitesse limite de la solution $u$ de condition initiale $u(x,0)=\mathbf{1}_{x<0}$ est égale à 2, on va construire une sous-solution et une sur-solution de \eqref{eq:ReaDi} qui ont toutes les deux une vitesse limite faible arbitrairement proche de 2. Par le principe de comparaison, $u$ aura alors une vitesse limite faible égale à 2.
\paragraph{Construction d'une sur-solution:} \ \\
Soit $v(x,t) = v_c(x,t)e^t$ où $v_c$ est la solution de l’équation de la chaleur avec condition initiale $v_c(x,0) =  \mathbf{1}_{x<0}$. Alors $v$ est $C^{1,2}$ et $\forall t > 0 $: \\
\begin{align*}
\dt v - \Delta v &= e^tv_c + e^t(\dt v_c - \Delta v_c) \\
&= e^tv_c \\
&= v \\
&\geq F(v) = v-v^2
\end{align*}
Donc $v$ est une sur-solution, de donnée initiale égale a $u$. Le théorème de comparaison peut donc s'appliquer et donne $u\leq v$, $\forall x, t >0$. Il faut alors montrer que $v$ satisfait la définition de vitesse limite faible pour la vitesse $s^* =2$. La preuve de ce point peut se trouver dans %Citer These Emeric Bouin
\paragraph{Construction d'une sous-solution:} \ \\
Soit $\epsilon > 0 $, posons $s_\epsilon = 2 - \epsilon $. Les solutions de la linéarisation de \eqref{eq:FWS} autour du point d’équilibre 0 sont oscillantes et s’écrivent de la forme $h(y)= A\exp(Ry+C)\cos(\omega y+C)$ où \\ $R + i\omega = -(1-\frac{\epsilon}{2})(1+i\sqrt{1-(1-\frac{\epsilon}{2})^2})$, $y= x- s_\epsilon t$. On va choisir $C = \frac{\pi + \omega}{2}$ et l'on va tronquer cette fonction pour obtenir une fonction $w(x,t)= h(x-s_\epsilon t )$ telle que:
\[ w(x,0)  = \begin{cases} h(x) \ \ \forall x \in [-\frac{1}{2} - \pi, -\frac{1}{2}] \\ 0 \text{    sinon}\end{cases}
\]
On peut alors choisir A tel que $h(y)<1$ ce qui donne $w(x,0)< \mathbf{1}_{x<0} \ \forall x$.\\
 La vitesse limite de $w$ est de plus bien $s_\epsilon$.\\\
Il reste alors à montrer que $w$ est une sous-solution de \eqref{eq:ReaDi}, ce qui nécessite quelques calculs effectués dans %Citer These Emeric Bouin
.\\
Par le théorème de comparaison, on a donc $w\leq u $, $\forall x, t>0$.
\\
\ \\
On a alors $w\leq u\leq v$ , donc $ 2-\epsilon \leq s^* \leq 2 $, ceci $\forall \epsilon$ donc $s^* =2$. 
\ifdefined\COMPLETE
\else
\end{document}
\fi