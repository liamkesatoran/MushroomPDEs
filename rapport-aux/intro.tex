\ifdefined\COMPLETE
\else
\documentclass[11pt]{article}
\usepackage[french, english]{babel}
\usepackage[utf8]{inputenc}
\usepackage{graphicx}
\usepackage{framed}
\usepackage[normalem]{ulem}
\usepackage{amsmath}
\usepackage{amsthm}
\usepackage{amssymb}
\usepackage{amsfonts}
\usepackage{enumerate}
\usepackage{import}
\usepackage[top=1 in,bottom=1in, left=1 in, right=1 in]{geometry}
\usepackage{listingsutf8}
\usepackage{color}
\usepackage{float}
\usepackage{graphicx}
\usepackage{subcaption}
\usepackage[toc,page]{appendix}
\usepackage{multicol}
\usepackage{wrapfig}
\usepackage{sidecap}

\floatstyle{boxed} 
\restylefloat{figure}
\definecolor{mygreen}{rgb}{0,0.6,0}
\definecolor{mygray}{rgb}{0.5,0.5,0.5}
\definecolor{mymauve}{rgb}{0.58,0,0.82}
\newcommand{\dt}{\partial_t}
\newcommand{\Tl}{\frac{T}{\lambda}}
\theoremstyle{definition}
\newtheorem{definition}{Définition}[section]
\DeclareMathOperator*{\argmax}{arg\,max}
\DeclareMathOperator*{\argmin}{arg\,min}
 


\lstset{ 
  backgroundcolor=\color{white},   % choose the background color; you must add \usepackage{color} or \usepackage{xcolor}; should come as last argument
  basicstyle=\footnotesize,        % the size of the fonts that are used for the code
  breakatwhitespace=false,         % sets if automatic breaks should only happen at whitespace
  breaklines=true,                 % sets automatic line breaking
  captionpos=b,                    % sets the caption-position to bottom
  commentstyle=\color{mygreen},    % comment style
  deletekeywords={...},            % if you want to delete keywords from the given language
  escapeinside={\%*}{*)},          % if you want to add LaTeX within your code
  extendedchars=true,              % lets you use non-ASCII characters; for 8-bits encodings only, does not work with UTF-8
  firstnumber=1000,                % start line enumeration with line 1000
  frame=single,	                   % adds a frame around the code
  keepspaces=true,                 % keeps spaces in text, useful for keeping indentation of code (possibly needs columns=flexible)
  keywordstyle=\color{blue},       % keyword style
  language=Python,                 % the language of the code
  morekeywords={*,...},            % if you want to add more keywords to the set
  numbers=left,                    % where to put the line-numbers; possible values are (none, left, right)
  numbersep=5pt,                   % how far the line-numbers are from the code
  numberstyle=\tiny\color{mygray}, % the style that is used for the line-numbers
  rulecolor=\color{black},         % if not set, the frame-color may be changed on line-breaks within not-black text (e.g. comments (green here))
  showspaces=false,                % show spaces everywhere adding particular underscores; it overrides 'showstringspaces'
  showstringspaces=false,          % underline spaces within strings only
  showtabs=false,                  % show tabs within strings adding particular underscores
  stepnumber=2,                    % the step between two line-numbers. If it's 1, each line will be numbered
  stringstyle=\color{mymauve},     % string literal style
  tabsize=2,	                   % sets default tabsize to 2 spaces
  title=\lstname                   % show the filename of files included with \lstinputlisting; also try caption instead of title
}
\lstset{inputencoding=utf8/latin1}
\newcommand{\Dt}{\Delta t}
\newcommand{\Dx}{\Delta x}
 %file containing all the used libraries
\begin{document}
\fi

\section{Introduction}


Comment une information ou une rumeur peut être relayée et se propager au sein d'un réseau numérique ou social? Comment un champignon ou une plante envahit un milieu pour y chercher au mieux des substances nutritives? Comment des virus ou des microbes pathogènes peuvent se propager au sein d'une population humaine ou animale? Comment les marchandises, l'énergie, l'argent peuvent circuler de la "meilleure" façon au sein d'une économie? \\
Toutes ces questions semblent se référer à des problématiques sensiblement différentes et à priori étudiées dans des disciplines différentes. L'analyse de ces phénomènes est liée à la dynamique de propagation et diffusion au sein d'un réseau d'agents connectés.  Leur description mathématique repose alors sur des modèles très similaires, qui portent sur la description et la caractérisation d'un nombre croissant d'"individus" en évolution, ainsi que de leurs interactions au sein d'un réseau en expansion spatiale. \\
Le but de ce stage consistait d'une part en l'analyse mathématique des modèles proposés, et d'autre part à développer des approches numériques permettant de simuler et d'analyser la dynamique de ces réseaux. \\
La première section de ce rapport est un rappel de résultats classiques sur les équations de réaction diffusion et en particulier l’équation de Fisher-KPP.\\ La seconde section consiste à l'introduction et l'explication du modèle étudié, le modèle de croissance de réseaux dynamiques branchant, ainsi que l’étude et l'analyse de certaines de ces propriétés. \\ La troisième section expose le calcul de recherche de la vitesse d'onde pour les solutions progressives d'une approximation du modèle.\\ La quatrième section introduit les schémas numériques utilisés pour simuler le modèle approché, ainsi que l'analyse de ces schémas.\\ 
La cinquième expose alors les résultats de la simulation numérique ainsi que l'analyse de ces résultats: notamment, on obtient dans cette partie une validation pour la vitesse d'onde obtenue dans la troisième section. \\
La sixième section revient alors sur le modèle complet (non simplifié) et expose comment retrouver la vitesse d'onde dans ce cas.\\
La septième et dernière section valide ensuite le résultat de la section 6 sur différentes simulations.









\ifdefined\COMPLETE
\else
\end{document}
\fi