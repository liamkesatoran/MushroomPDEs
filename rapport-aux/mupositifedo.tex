\iffalse
\documentclass[11pt]{article}
\usepackage[french, english]{babel}
\usepackage[utf8]{inputenc}
\usepackage{graphicx}
\usepackage{framed}
\usepackage[normalem]{ulem}
\usepackage{amsmath}
\usepackage{amsthm}
\usepackage{amssymb}
\usepackage{amsfonts}
\usepackage{enumerate}
\usepackage{import}
\usepackage[top=1 in,bottom=1in, left=1 in, right=1 in]{geometry}
\usepackage{listingsutf8}
\usepackage{color}
\usepackage{float}
\usepackage{graphicx}
\usepackage{subcaption}
\usepackage[toc,page]{appendix}
\floatstyle{boxed} 
\restylefloat{figure}
\definecolor{mygreen}{rgb}{0,0.6,0}
\definecolor{mygray}{rgb}{0.5,0.5,0.5}
\definecolor{mymauve}{rgb}{0.58,0,0.82}
\newcommand{\dt}{\partial_t}
\newcommand{\Tl}{\frac{T}{\lambda}}
\theoremstyle{definition}
\newtheorem{definition}{Définition}[section]
\begin{document}
\fi


\subsection{Propriétés de l'EDO ``KPP avec mémoire"}
\newtheorem{lemma}{Lemme}
Soit $(\mu,\rho,C)$ vérifiant le système d’équations suivant:
\begin{equation} \left\{
                \begin{array}{ll}
                   \dt\mu  = f(C)(\mu + \rho) -\mu\rho\\
                 \dt\rho=  F_0 \mu \\
                  \dt C = -b\rho C
                \end{array}
              \right.
\end{equation} avec $f(0)=0$
On s'interesse au comportement de $(\mu,\rho,C)$ sur $\mathbb{R}^+$
\begin{lemma}C est de signe constant.\\
En effet on a $C(t)= C(0)\exp(-b\int_{0}^{t}\rho(s)ds)$.
\end{lemma}

\begin{lemma}Soit $(\mu,\rho,C)$ tel que $(\mu(0),\rho(0))> (0,0)$ (les deux positifs, au moins un non nul), $C(0)>0$.\\  Alors $\mu(t)\geq 0$ $\forall t>0$
\end{lemma}
\begin{proof}
Supposons par l'absurde que $\mu$ devient négatif alors soit $t^*= \min(t>0/ \mu(t)<0)$. Alors: \\
$\mu(t)\geq 0$ $\forall t \leq t^*$\\
$\dt\mu(t^*) \leq 0$ par définition de $t^*$. (Sinon $\mu(t^*+\epsilon)>0$ $\forall \epsilon <<1$)\\
$\rho(t)>0$ $\forall t\leq t^*$ car $\dt\rho=  F_0 \mu$ et $F_0>0$\\
$\dt\mu(t^*) = f(C(t^*))\rho(t^*) > 0$ ce qui est en contradiction avec la deuxième affirmation.
\end{proof}
Dans la suite on se place dans le cas $(\mu(0),\rho(0))> (0,0)$, $C(0)>0$:
\begin{lemma} $\rho$ est croissante car $\dt\rho=  F_0 \mu \geq 0$. \end{lemma}



\newpage

\begin{lemma}$\lim_{+\infty}C(t) = 0$ \end{lemma}
\begin{proof} $(\mu(0),\rho(0))> (0,0)$ et $\dt\rho=  F_0 \mu$ impliquent qu'il existe un $t_0$ tel que $\mu(t_0)>0$. Comme $\rho$ est croissante $\forall t\geq t_0$, $\rho(t) \geq \rho(t_0)$.\\Donc $\forall t\geq t_0$, $C(t)= C(0)\exp(-b\int_{0}^{t}\rho(s)ds) \leq C_{ste}e^{-b \rho(t_0)t}$.
\end{proof}

\begin{lemma} Si $f^2$ est intégrable en 0, alors $\mu$ est bornée
\end{lemma}
\begin{proof} $\forall t\geq 0$, $C(t)\leq C_{ste}e^{-b \rho(t_0)t} $ Soit $\epsilon(t)=f(C_{ste}e^{-b \rho(t_0)t} )$, On a $\forall t\geq 0$,\\ $\dt\mu  = C(\mu + \rho) -\mu\rho\ \leq \epsilon(\mu + \rho) -\mu\rho$ \\
Étudions la fonction $F$ : $x,y \mapsto \epsilon(x + y) -xy$ sur $\mathbb{R}^+ \times \mathbb{R}^+$\\
$\nabla F = (\epsilon-y, \epsilon-x)$ donc il y a un point critique en $\epsilon,\epsilon$.\\
$H(F) = \begin{pmatrix}
   0 & -1 \\
   -1 & 0 
\end{pmatrix}$ donc $(x,y)H(F)\begin{pmatrix} x \\ y \end{pmatrix} = -2xy < 0$ sur $\mathbb{R}^+ \times \mathbb{R}^+$ donc $\epsilon,\epsilon$ est un maximum local: comme c'est le seul point critique, il est en fait global.\\
On a donc $F(x,y)\leq F(\epsilon,\epsilon) = \epsilon^2 $ $\forall x,y$ dans $\mathbb{R}^+ \times \mathbb{R}^+$\\
Conclusion:$\forall t>T$,  $\dt\mu \leq F(\rho,\mu) \leq  \epsilon^2 \leq f(C_{ste}e^{-2b \rho(t_0)t})$. \\En particulier $\dt\mu$ est majoré par une fonction intégrable en $+\infty$ (car $f^2$ est intégrable en 0 et par changement de variable) donc $\mu$ est majoré, et minoré par $0$, donc $\mu$ est bornée.
\end{proof}
Dans la suite, on suppose $f^2$ intégrable en 0:
\begin{lemma} $\lim_{+\infty} \mu(t) = 0$
\end{lemma}
\begin{proof}
$\dt\mu \leq f(C_{ste}e^{-2b \rho(t_0)t})$ montre que $\limsup _{+\infty}\dt\mu \leq 0$.\\
Par l'absurde, si on a $\limsup _{+\infty}\dt\mu <0$ alors $\mu$ n'est pas bornée.\\
On a donc $\limsup_{+\infty}f(C)(\mu + \rho) -\mu\rho = 0$.
Donc $\limsup_{+\infty} f(C)(\mu) + \rho ( f(C)-\mu) = 0$.\\ $\mu$ est bornée et $\lim_{+\infty}f(C)=0$ donc $\limsup_{+\infty} f(C)(\mu) =0$ .\\
On a donc $0=  \limsup_{+\infty} \rho (f(C)-\mu)$. $\rho$ étant croissante positive, $\rho$ admet une limite dans  $\overline{\mathbb{R^+}}$ mais $0=  \limsup_{+\infty} \rho (f(C)-\mu)$ donc $\lim_{+\infty}\mu =0$.
\end{proof} 

\begin{lemma}$\lim_{+\infty} \rho(t)= \rho_\infty < +\infty$  \end{lemma}
\begin{proof}On a:\\
$\int_0^{+\infty}\mu(t)dt \leq \int_0^{+\infty}\int_t^{+\infty}f^2(C_{ste}e^{-2b \rho(t_0)s})\ dsdt$\\
Faisons le changement de variable $p = C_{ste}e^{-2b \rho(t_0)s}$, $dp = -2b \rho(t_0)pds$:\\
$\int_0^{+\infty}\mu(t)dt \leq A\int_0^{+\infty}\int_0^{C_{ste}e^{-2b \rho(t_0)t}}f^2(p)\ dpdt$\\
Faisons le changement de variable $k = C_{ste}e^{-2b \rho(t_0)t}$, $dk = -2b \rho(t_0)pdt$ dans la deuxième intégrale:\\
$\int_0^{+\infty}\mu(t)dt \leq B\int_0^{1}\int_0^{k}f^2(p)\ dp dk < +\infty $\\
$\mu$ est donc intégrable sur $\mathbb{R}^+$. Or $ \dt\rho=  F_0 \mu$. Donc $\lim_{+\infty} \rho(t)= \rho_\infty < +\infty$ 
\end{proof}
%\end{document}