\iffalse
\documentclass[11pt]{article}
\usepackage[french, english]{babel}
\usepackage[utf8]{inputenc}
\usepackage{graphicx}
\usepackage{framed}
\usepackage[normalem]{ulem}
\usepackage{amsmath}
\usepackage{amsthm}
\usepackage{amssymb}
\usepackage{amsfonts}
\usepackage{enumerate}
\usepackage{import}
\usepackage[top=1 in,bottom=1in, left=1 in, right=1 in]{geometry}
\usepackage{listingsutf8}
\usepackage{color}
\usepackage{float}
\usepackage{graphicx}
\usepackage{subcaption}
\usepackage[toc,page]{appendix}
\floatstyle{boxed} 
\restylefloat{figure}
\definecolor{mygreen}{rgb}{0,0.6,0}
\definecolor{mygray}{rgb}{0.5,0.5,0.5}
\definecolor{mymauve}{rgb}{0.58,0,0.82}
\newcommand{\dt}{\partial_t}
\newcommand{\Tl}{\frac{T}{\lambda}}
\theoremstyle{definition}
\newtheorem{definition}{Définition}[section]
\begin{document}
\fi


\subsection{Propriétés de l'EDO ``KPP avec mémoire"}
\newtheorem{lemma}{Lemme}
Soit $(\mu,\rho,C)$ vérifiant le système d’équations suivant:
\begin{equation} \left\{
                \begin{array}{ll}
                   \dt\mu  = f(C)(\mu + \rho) -\mu\rho\\
                 \dt\rho=  F_0 \mu \\
                  \dt C = -b\rho C
                \end{array}
              \right.
\end{equation} avec $f(0)=0$
On s'interesse au comportement de $(\mu,\rho,C)$ sur $\mathbb{R}^+$
\begin{lemma}C est de signe constant.\\
En effet on a $C(t)= C(0)\exp(-b\int_{0}^{t}\rho(s)ds)$.
\end{lemma}

\begin{lemma}Soit $(\mu,\rho,C)$ tel que $(\mu(0),\rho(0))> (0,0)$ (les deux positifs, au moins un non nul), $C(0)>0$.\\  Alors $\mu(t)\geq 0$ $\forall t>0$
\end{lemma}
\begin{proof}
Supposons par l'absurde que $\mu$ devient négatif alors soit $t^*= \min(t>0/ \mu(t)<0)$. Alors: \\
$\mu(t)\geq 0$ $\forall t \leq t^*$\\
$\dt\mu(t^*) \leq 0$ par définition de $t^*$. (Sinon $\mu(t^*+\epsilon)>0$ $\forall \epsilon <<1$)\\
$\rho(t)>0$ $\forall t\leq t^*$ car $\dt\rho=  F_0 \mu$ et $F_0>0$\\
$\dt\mu(t^*) = f(C(t^*))\rho(t^*) > 0$ ce qui est en contradiction avec la deuxième affirmation.
\end{proof}
Dans la suite on se place dans le cas où $(\mu(0),\rho(0))> (0,0)$, $C(0)>0$:
\begin{lemma} $\rho$ est croissante car $\dt\rho=  F_0 \mu \geq 0$. En particulier $\rho$ est positive \end{lemma}



\begin{lemma}$C$ est décroissante et $\underset{t\to+\infty} \lim C(t) = 0$ \end{lemma}
\begin{proof}$\rho$ est positive donc $C$ est décroissante.\\
 $(\mu(0),\rho(0))> (0,0)$ et $\dt\rho=  F_0 \mu$ impliquent qu'il existe un $t_0$ tel que $\rho(t_0)>0$.\\ Comme $\rho$ est croissante $\forall t\geq t_0$, $\rho(t) \geq \rho(t_0)$.\\Donc $\forall t\geq t_0$, $0<C(t)= C(0)\exp(-b\int_{0}^{t}\rho(s)ds) \leq C_{ste}e^{-b \rho(t_0)t}\underset{t\to+\infty}{\longrightarrow}0$ \\
Donc $\underset{t\to+\infty} \lim C(t) = 0$.
\end{proof}

\newpage 


\begin{lemma} Si $f$ est croissante et  $f(C)=\mathcal{O}_{C\to 0}(C)$ alors $\mu$ est bornée. \end{lemma}
\begin{proof} On a $\dt\mu  = f(C)(\mu + \rho) -\mu\rho \leq f(C)\mu + f(C)\rho$.\\
Soit $\phi(t)= f(C(t))\rho(t)$.\\
Comme $f(C)=\mathcal{O}_{C\to 0} (C)$, $C$ décroît vers 0, $f$ est croissante et $C(t)\leq C_{ste}e^{-b \rho(t_0)t}$: $f(C)$ est intégrable \\
 Comme $f(C)=\mathcal{O}_{C\to 0} (C)$ et $C$ décroît vers 0, $\exists A,T$ \ $\forall t > T, $  $ \phi(t) < AC(t)\rho(t)$.\\
On a donc $\int_T^{+\infty}\phi(t)\ dt< \int_T^{+\infty}AC\rho \ dt  =  -\frac{A}{b}\int_T^{+\infty}\dt C = \frac{A}{b}C(T)$.\\
$\phi$ est donc intégrable et $\dt\mu  \leq f(C)\mu + \phi$.\\
Par le lemme de Gronwall:\\
$\mu(t) \leq \mu(0) + \int_0^t \phi(s) \ ds+\int_0^t \phi(s)f(C)(s)\exp(\int_s^t f(C)(u)du)\ ds \\
 \leq \mu(0) + \int_0^{+\infty} \phi(s) \ ds+ \int_0^t \phi(s)f(C)(s)\exp(\int_0^{+\infty} f(C)(u)du)\ ds \\
 \leq  \mu(0) + \int_0^{+\infty} \phi(s) \ ds+ \exp(\int_0^{+\infty} f(C)(u)du)  \int_0^t \phi(s)f(C)(s) \ ds$\\
$f(C)$ est bornée et  $\phi$ est intégrable donc $f(C)\phi$ est intégrable.\\
On a donc:
$\mu(t) \leq  \mu(0) + \int_0^{+\infty} \phi(s) \ ds+ \exp(\int_0^{+\infty} f(C)(u)du)  \int_0^{+\infty} \phi(s)f(C)(s) \ ds$ $\forall t$
\end{proof}

Dans la suite on se place dans le cas où $f$ est croissante et  $f(C)=\mathcal{O}_{C\to 0}(C)$
\begin{lemma} $\mu$ décroît à partir d'un certain temps et $\underset{t\to+\infty} \lim \mu(t) = 0$ \end{lemma}
\begin{proof} 
On a $\dt \mu = f(C)(\mu + \rho) -\mu\rho \leq \mu (f(C)-\rho(t_0)) + f(C)\rho $.\\
Donc $\forall t>t_0$ $\frac{\dt \mu}{\mu}  \leq f(C)-\rho(t_0) +\frac{f(C)\rho}{\mu}$. \\
On a $\underset{t\to+\infty} \lim f(C) =0$. De plus $C$ décroît vers 0 donc $\underset{t\to+\infty} \lim \dt C = 0$ donc $\underset{t\to+\infty} \lim f(C)\rho =0$. \\
Ainsi $\exists T$ / $\forall t>T$ : $ \dt \log\mu < 0$ et donc, $\mu $ décroît à partir de $T$.\\
$\mu$ décroit à partir de $T$ et $\mu$ est bornée donc $\mu$ a une limite $\ell$ et $\underset{t\to+\infty} \lim \dt \mu = 0$.\\
On a donc $0 = \underset{t\to+\infty} \lim f(C)(\mu+\rho)-\mu\rho = \underset{t\to+\infty} \lim -\mu\rho  \leq -l\rho(t_0)$ donc $\ell =0$.
\end{proof}
\begin{lemma} $\underset{t\to+\infty} \lim \rho(t) =\rho_\infty < +\infty $ \end{lemma}
\begin{proof}
On a $ \mu(t) = - \int_t^{+\infty}\dt \mu \ ds = - \int_t^{+\infty} f(C)(\mu+\rho)-\mu\rho \ ds$.\\
Or $f(C)$ est intégrable (c.f. preuve du lemme 5)et $\mu$ est bornée donc $f(C)\mu$ est intégrable.\\
De même $ f(C)\rho$ est intégrable (c.f. preuve du lemme 5).\\
On peut donc en conclure que $\mu\rho$ est intégrable.\\
Or $\mu\rho= F_0\rho\dt\rho = \frac{F_0}{2}\dt \rho ^2$. $t \mapsto \dt\rho ^2$ est donc intégrable. \\Ainsi $\rho^2$ possède une limite finie et donc $\rho$ aussi.
\end{proof}
%\end{document}