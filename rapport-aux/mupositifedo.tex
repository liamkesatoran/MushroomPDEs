\ifdefined\COMPLETE
\else
\documentclass[11pt]{article}
\usepackage[french, english]{babel}
\usepackage[utf8]{inputenc}
\usepackage{graphicx}
\usepackage{framed}
\usepackage[normalem]{ulem}
\usepackage{amsmath}
\usepackage{amsthm}
\usepackage{amssymb}
\usepackage{amsfonts}
\usepackage{enumerate}
\usepackage{import}
\usepackage[top=1 in,bottom=1in, left=1 in, right=1 in]{geometry}
\usepackage{listingsutf8}
\usepackage{color}
\usepackage{float}
\usepackage{graphicx}
\usepackage{subcaption}
\usepackage[toc,page]{appendix}
\usepackage{multicol}
\usepackage{wrapfig}
\usepackage{sidecap}

\floatstyle{boxed} 
\restylefloat{figure}
\definecolor{mygreen}{rgb}{0,0.6,0}
\definecolor{mygray}{rgb}{0.5,0.5,0.5}
\definecolor{mymauve}{rgb}{0.58,0,0.82}
\newcommand{\dt}{\partial_t}
\newcommand{\Tl}{\frac{T}{\lambda}}
\theoremstyle{definition}
\newtheorem{definition}{Définition}[section]
\DeclareMathOperator*{\argmax}{arg\,max}
\DeclareMathOperator*{\argmin}{arg\,min}
 


\lstset{ 
  backgroundcolor=\color{white},   % choose the background color; you must add \usepackage{color} or \usepackage{xcolor}; should come as last argument
  basicstyle=\footnotesize,        % the size of the fonts that are used for the code
  breakatwhitespace=false,         % sets if automatic breaks should only happen at whitespace
  breaklines=true,                 % sets automatic line breaking
  captionpos=b,                    % sets the caption-position to bottom
  commentstyle=\color{mygreen},    % comment style
  deletekeywords={...},            % if you want to delete keywords from the given language
  escapeinside={\%*}{*)},          % if you want to add LaTeX within your code
  extendedchars=true,              % lets you use non-ASCII characters; for 8-bits encodings only, does not work with UTF-8
  firstnumber=1000,                % start line enumeration with line 1000
  frame=single,	                   % adds a frame around the code
  keepspaces=true,                 % keeps spaces in text, useful for keeping indentation of code (possibly needs columns=flexible)
  keywordstyle=\color{blue},       % keyword style
  language=Python,                 % the language of the code
  morekeywords={*,...},            % if you want to add more keywords to the set
  numbers=left,                    % where to put the line-numbers; possible values are (none, left, right)
  numbersep=5pt,                   % how far the line-numbers are from the code
  numberstyle=\tiny\color{mygray}, % the style that is used for the line-numbers
  rulecolor=\color{black},         % if not set, the frame-color may be changed on line-breaks within not-black text (e.g. comments (green here))
  showspaces=false,                % show spaces everywhere adding particular underscores; it overrides 'showstringspaces'
  showstringspaces=false,          % underline spaces within strings only
  showtabs=false,                  % show tabs within strings adding particular underscores
  stepnumber=2,                    % the step between two line-numbers. If it's 1, each line will be numbered
  stringstyle=\color{mymauve},     % string literal style
  tabsize=2,	                   % sets default tabsize to 2 spaces
  title=\lstname                   % show the filename of files included with \lstinputlisting; also try caption instead of title
}
\lstset{inputencoding=utf8/latin1}
\newcommand{\Dt}{\Delta t}
\newcommand{\Dx}{\Delta x}
 %file containing all the used libraries
\begin{document}
\fi



\subsection{Propriétés de l'équation de réaction associée à ``KPP avec mémoire"}
\newtheorem{lemma}{Lemme}
Soit $(\mu,\rho,C)$ vérifiant le système d’équations suivant:
\begin{equation} \left\{
                \begin{array}{ll}
                   \dt\mu  = f(C)(\mu + \rho) -\mu\rho\\
                 \dt\rho=  F_0 \mu \\
                  \dt C = -b\rho C
                \end{array}
              \right.
\end{equation} avec $f(0)=0$. \\
Ce système correspond au systême ``KPP avec mémoire" sans le terme de diffusion.
On s’intéresse au comportement de $(\mu,\rho,C)$ sur $\mathbb{R}^+$ : 
\begin{lemma}$C$ est de signe constant.\\
En effet on a $C(t)= C(0)\exp(-b\int_{0}^{t}\rho(s)ds)$.
\end{lemma}

\begin{lemma}Soit $(\mu,\rho,C)$ tel que $(\mu(0),\rho(0))> (0,0)$ (les deux positifs, au moins un non nul), $C(0)>0$.\\  Alors $\mu(t)\geq 0$ $\forall t>0$.
\end{lemma}
\begin{proof}
Supposons par l'absurde que $\mu$ devient négatif alors soit $t^*= \inf(t>0/ \mu(t)<0)$. Alors: \\
$\mu(t)\geq 0$ $\forall t \leq t^*$\\
$\dt\mu(t^*) \leq 0$ par définition de $t^*$. (Sinon $\mu(t^*+\epsilon)>0$ $\forall \epsilon <<1$)\\
$\rho(t)>0$ $\forall t\leq t^*$ car $\dt\rho=  F_0 \mu$ et $F_0>0$\\
$\dt\mu(t^*) = f(C(t^*))\rho(t^*) > 0$ ce qui est en contradiction avec la deuxième affirmation.
\end{proof}
Dans la suite on se place dans le cas où $(\mu(0),\rho(0))> (0,0)$, $C(0)>0$:
\begin{lemma}
 $\rho$ est croissante car $\dt\rho=  F_0 \mu \geq 0$. En particulier $\rho$ est positive.
\end{lemma}



\begin{lemma}$C$ est décroissante et $\underset{t\to+\infty} \lim C(t) = 0$ \end{lemma}
\begin{proof}$\rho$ est positive donc $C$ est décroissante.\\
 $(\mu(0),\rho(0))> (0,0)$ et $\dt\rho=  F_0 \mu$ impliquent qu'il existe un $t_0$ tel que $\rho(t_0)>0$.\\ Comme $\rho$ est croissante $\forall t\geq t_0$, $\rho(t) \geq \rho(t_0)$.\\Donc $\forall t\geq t_0$, $0<C(t)= C(0)\exp(-b\int_{0}^{t}\rho(s)ds) \leq C_{ste}e^{-b \rho(t_0)t}\underset{t\to+\infty}{\longrightarrow}0$ \\
Donc $\underset{t\to+\infty} \lim C(t) = 0$.
\end{proof}

\newpage 


\begin{lemma} Si $f$ est croissante et  $\int_0^1 \frac{f(x)}{x} dx < \infty $ alors $\mu$ est bornée. \end{lemma}
\begin{proof} On a $\dt\mu  = f(C)(\mu + \rho) -\mu\rho \leq f(C)\mu + f(C)\rho$.
\\
Montrons que $f(C)$ est intégrable:
\\
$C(t)\leq C_{ste}e^{-b \rho(t_0)t}$ et $f$ est croissante donc $\int_0^\infty f(C)dt \leq \int_0^\infty f(C_{ste}e^{-b \rho(t_0)t})dt$.
\\
Soit le changement de variable $u= C_{ste}e^{-b \rho(t_0)t}$, $du = -b\rho(t_0)u\ dt$:
\\
$\int_0^\infty f(C_{ste}e^{-b \rho(t_0)t})dt = \frac{1}{b\rho(t_0)} \int_0^1 \frac{f(u)}{u} du < \infty$ car $\int_0^{C_{ste}} \frac{f(x)}{x} dx < \infty $ donc $f(C)$ est intégrable.
\\
Montrons que $\phi=f(C)\rho$ est intégrable:\\
Effectuons le changement de variable $u = C$, $du = -b\rho u \ dt$ dans $\int_0^\infty f(C) \rho \ dt$:
\\
$\int_0^\infty f(C) \rho \ dt = \frac{1}{b\rho(t_0)} \int_0^{C_{ste}} \frac{f(u)}{u} du  < \infty$ car $\int_0^{C_{ste}} \frac{f(x)}{x} dx < \infty $ donc $\phi = f(C)\rho$ est intégrable.
\\
On a $\dt\mu  \leq f(C)\mu + \phi$.\\
Par le lemme de Gronwall:\\
$\mu(t) \leq \mu(0) + \int_0^t \phi(s) \ ds+\int_0^t \phi(s)f(C)(s)\exp(\int_s^t f(C)(u)du)\ ds \\
 \leq \mu(0) + \int_0^{+\infty} \phi(s) \ ds+ \int_0^t \phi(s)f(C)(s)\exp(\int_0^{+\infty} f(C)(u)du)\ ds \\
 \leq  \mu(0) + \int_0^{+\infty} \phi(s) \ ds+ \exp(\int_0^{+\infty} f(C)(u)du)  \int_0^t \phi(s)f(C)(s) \ ds$\\
$f(C)$ est bornée et  $\phi$ est intégrable donc $f(C)\phi$ est intégrable.\\
On a donc:
$\mu(t) \leq  \mu(0) + \int_0^{+\infty} \phi(s) \ ds+ \exp(\int_0^{+\infty} f(C)(u)du)  \int_0^{+\infty} \phi(s)f(C)(s) \ ds$ $\forall t$
\end{proof}

Dans la suite on se place dans le cas où $f$ est croissante et   $\int_0^1 \frac{f(x)}{x} dx < \infty $
\begin{lemma} $\underset{t\to+\infty} \lim \mu = 0$ \  et $\underset{t\to+\infty} \lim \rho =\rho_\infty < +\infty $ \end{lemma}
\begin{proof}
$\mu$ est bornée, soit $\mu_n$ une suite extraite de la fonction $\mu$ qui tend vers $\ell$.\\
On a $ \ell- \mu(t) = \underset{n\to+\infty} \lim \int_t^{t_n}\dt \mu \ ds = \underset{n\to+\infty} \lim \int_t^{t_n} f(C)(\mu+\rho)-\mu\rho \ ds$.\\
Or $f(C)$ est intégrable (c.f. preuve du lemme 5) et $\mu$ est bornée donc $f(C)\mu$ est intégrable.\\
De même $ f(C)\rho$ est intégrable (c.f. preuve du lemme 5).\\
On a donc $\underset{n\to+\infty} \lim \int_t^{t_n} \mu\rho =  \int_t^{+\infty}(f(C)\mu + f(C)\rho) \ dt \ + \ell -\mu(t) $\\
Or $\mu\rho= F_0\rho\dt\rho = \frac{F_0}{2}\dt \rho ^2$ donc $\int_t^{t_n} \mu\rho \ dt= \frac{F_0}{2}( \rho(t_n) ^2 -\rho(t)^2)$. \\
Or $\rho$ est croissante donc a une limite dans $[0,+\infty]. \\
$Ainsi $\ell$ est déterminée entièrement par la limite de $\rho$ et ne dépend pas de la suite extraite.\\
Par critère séquentiel $\mu$ a une limite $\ell$ qui est finie car $\mu$ est bornée.\\
Mais alors  $\underset{t\to+\infty} \lim \frac{F_0}{2}( \rho(t) ^2 -\rho(T)^2) = \int_T^{\infty}(f(C)\mu + f(C)\rho) \ dt \ + \ell -\mu(T) < \infty$.\\
Donc $\rho^2$ a une limite finie et donc $\rho$ aussi.\\
Comme $\mu = \frac{\dt \rho}{F_0}$ et $\rho$ a une limite finie et $\mu$ aussi, $\mu$ tend nécessairement vers 0. 	 
\end{proof}


\ifdefined\COMPLETE
\else
\end{document}
\fi